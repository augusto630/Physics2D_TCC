\documentclass[12pt,a4paper,oneside]{book} %{livro} para o livro é necessário alterar o arquivo fonte

\usepackage[brazilian]{babel}
\usepackage[T1]{fontenc}
\usepackage{ae}
\usepackage{indentfirst}
\usepackage[ansinew]{inputenc} 
\usepackage{amsmath,amssymb,amsfonts,array,graphicx}
\usepackage{appendix}
\usepackage{fancyhdr}
\usepackage{color}
\usepackage{lipsum}
\usepackage{epstopdf} %incluindo .eps para compilar em latex




\usepackage{color}
	\definecolor{UELSymbol}{cmyk}{1,0,1,0.2}
	\definecolor{UNIFILSymbol}{cmyk}{0,0.425,0.845,0.012}
	\definecolor{citecollor}{cmyk}{1,0,0,0}%para internet
	\definecolor{filecollor}{cmyk}{1,1,1,0}
	\definecolor{linkcollor}{rgb}{0,0,1}
	\definecolor{urlcollor}{rgb}{0.2,0.2,1}
	\definecolor{white}{rgb}{1,1,1} %para a página do apêndice e página em branco

%\definecolor{citecollor}{rgb}{0,0,0} %para impressão
%	\definecolor{filecollor}{rgb}{0,0,0}
%	\definecolor{linkcollor}{rgb}{0,0,0}
%	\definecolor{urlcollor}{rgb}{0,0,0}
	

%\usepackage[square,comma,numbers,super,compress]{natbib} 
\usepackage[square,comma,numbers,compress]{natbib} 


\usepackage[pdftex]{hyperref}
	\hypersetup{
							colorlinks=true,
							citecolor=citecollor,
							filecolor=filecollor,
							linkcolor=linkcollor,
							urlcolor=urlcollor,
							pageanchor=false}


\usepackage[font=small,labelfont=bf,labelsep=endash,skip=3pt,figurename=Figura,tablename=Tabela]{caption}
%	\captionsetup[table]{singlelinecheck=false}
	\captionsetup[figure]{justification=centerlast,singlelinecheck=true}


\usepackage[font=small]{subcaption}
		\captionsetup{subrefformat=parens}
			
\usepackage[final]{pdfpages}

\usepackage{setspace} 		
	\onehalfspacing 

\renewcommand{\arraystretch}{1.0} 

\setlength{\parindent}{3cm} 
\usepackage[includehead=true, top=2.0cm, left=3.0cm, bottom=2.0cm, right=2.0cm]{geometry} 

\setlength{\headheight}{15pt}
																																													
\begin{document}


\newcommand{\Nn}{\textbf{n}}
\newcommand{\aA}{\textbf{A}}
\newcommand{\bB}{\textbf{B}}
\newcommand{\Xx}{\textbf{x}}
\newcommand{\Yy}{\textbf{y}}
\newcommand{\Rr}{\textbf{r}}
\newcommand{\opc}{\mathbb{C}}
\newcommand{\opo}{\mathbb{O}}
\newcommand{\ops}{\mathbb{S}}
\newcommand{\opa}{\mathbb{A}}
\newcommand{\opt}{\mathbb{T}}


\newcommand{\ud}{\mathrm{d}}
 \newcommand{\sen}{\mathrm{sen\:}}	
  \newcommand{\degree}{\ensuremath{^\circ}}

\hyphenation{e-xem-plo} 
\hyphenation{trans-for-ma-se}
\hyphenation{a-poia-ram}
\hyphenation{cur-va-tu-ra}




\frontmatter
\pagenumbering{arabic}
\pagestyle{fancy}
\fancyhf{}
\lhead{}
\chead{}
\rhead{}
\lfoot{}
\cfoot{}
\rfoot{}
\renewcommand{\headrulewidth}{0pt}
\renewcommand{\footrulewidth}{0pt}

\addtocontents{toc}{\protect\thispagestyle{empty}} 
\addtocontents{lof}{\protect\thispagestyle{empty}} 
\addtocontents{lot}{\protect\thispagestyle{empty}} 

%início da capa


\thispagestyle{empty}
\vspace*{-20mm}
\begin{center}      %\begin{flushleft}
\includegraphics[scale=1.200]{LOGO/0_0_logo}
\end{center}        %\end{flushleft}
{\color{UNIFILSymbol}
\vspace{-6mm}
\noindent\rule[-1ex]{\textwidth}{5pt}}

\vspace{3mm}

\begin{center}
{\large  AUGUSTO CESAR NALIN RODRIGUES}
\end{center}

\vspace{42mm}


\begin{center}

\textbf{\large SIMULAÇÃO DE FÍSICA COM ENFASE EM COLISÃO DE CORPOS RÍGIDOS}

\end{center}


\vspace{119mm}

{\color{UNIFILSymbol}
\noindent\rule[-1ex]{\textwidth}{5pt}}

\vspace{0mm}

\begin{singlespace}
\begin{center}
Londrina\\
2014
\end{center}
\end{singlespace}

%fim da capa
\newpage


%inicio da folha de rosto


\thispagestyle{empty}
\setcounter{page}{1}
\begin{center}
{\large  FULANO CICLANO BELTRANO}

\vspace{9.3cm}

\textbf{\large TÍTULO DO SEU TRABALHO DE CONCLUSÃO ``SEU MOÇO´´}

\end{center}

\vspace{1.0cm}
\begin{singlespace}
\noindent
\begin{flushright}
\begin{minipage}[t]{9cm}

Trabalho de conclusão de curso apresentada ao, da Universidade \_\_\_\_\_, como requisito parcial à obtenção do título de\linebreak Bacharel em \_\_\_\_.

\vspace{0.5 cm}

Orientador: Prof\degree. Dr. Seu Fulano Ajudador Tabajara.

\end{minipage}
\end{flushright}
\end{singlespace}

\vspace{5.8cm}

\begin{singlespace}
\begin{center}
Londrina\\
2014
\end{center}
\end{singlespace}

%fim da folha de rosto.

\newpage

%inicio da folha de catalogação

%\includepdf[pages={1}]{Catalogue} %quando a tiver, substitua o arquivo com este nome que consta na pasta.


%fim da folha de catalogação


%início da folha de aprovação

\thispagestyle{empty}

\setcounter{page}{2}
\begin{center}
{\large  FULANO CICLANO BELTRANO}
\vspace{2.0cm}

\textbf{\large TÍTULO DO SEU TRABALHO DE CONCLUSÃO ``SEU MOÇO´´}

\end{center}

\vspace{2.0cm}
\begin{singlespace}
\noindent
\begin{flushright}
\begin{minipage}[t]{7cm}

Trabalho de conclusão de curso apresentada ao, da Universidade \_\_\_\_\_, como requisito parcial à obtenção do título de\linebreak Bacharel em \_\_\_\_.

\end{minipage}
\end{flushright}
\vspace{0.8 cm}
\noindent

\begin{center}
  \parbox{11cm}
  {
  \begin{center}
  \textbf{BANCA EXAMINADORA} \\
  \vspace{1.5cm}
  \rule{9cm}{.1mm}\\
    Prof. Dr. Seu Fulano Ajudador Tabajara\\Unifil -- Londrina -- PR \\
    \vspace{6mm}
  \rule{9cm}{.1mm} \\
    Prof. Dr. \textcolor{red}{Forever Alone} \\\textcolor{white}{DMU -- Derpland -- IT} \\
    \vspace{6mm}
  \rule{9cm}{.1mm} \\
    Profa. Dr. \textcolor{red}{Forever Alone} \\\textcolor{white}{DMU -- Derpland -- IT} \\
  \vspace{4mm}
  \rule{9cm}{.1mm} \\
    Prof. Dr. \textcolor{white}{Forever Alone} \\\textcolor{white}{DMU -- Derpland -- IT} \\
    \vspace{6mm}
  \rule{9cm}{.1mm} \\
    Prof. Dr. \textcolor{white}{Forever Alone} \\\textcolor{white}{DMU -- Derpland -- IT} \\
  \vspace{1.7cm}
  Londrina,\underline{\hspace{1cm}}de \underline{\hspace{3cm}} de 2014.
  \end{center}
  }
\end{center}
\end{singlespace}

%fim da folha de aprovação.

\newpage 
\vspace*{206mm}

%folha de dedicatória
\begin{flushright}
\begin{minipage}[!b]{8cm}
\emph{Esse trabalho é dedicado a... aqui pode ir uma lista ou não..}
\end{minipage}
\end{flushright}

\clearpage
\newpage 

%fim da dedicatória

%início dos agradecimentos

\begin{center} 
\textbf{AGRADECIMENTOS}
\end{center}
\vskip 1.0cm

\lipsum[1-3] %comente esse comando para tirar texto aleatório




%fim dos agradecimentos.

\newpage%inicio da epígrafe

\vspace*{180mm}

\begin{flushright}
\begin{minipage}[!b]{10cm}
\emph{O começo de todas as ciências é o espanto de as coisas serem o que são.}
\vspace{5mm}


\hfill (Aristóteles)


\end{minipage}
\end{flushright}


\clearpage %fim da epígrafe

\newpage

%início resumo

\noindent{GUTO, A. C. N. R. \textbf{Título do Trabalho de Conclusão seu Moço}. 2013. N\degree 100 f. Tese (Bacharelado em Ciências da Computação) -- Universidade do Templo Unico do Saber, 2013.}
\vskip 1.0cm

\begin{center} 
\textbf{RESUMO}
\end{center}
\vskip 0.5cm

\begin{noindent}

\lipsum[3-4]

\vskip 10mm

\noindent \textbf{Palavras-chave:}  

\end{noindent} 

%fim resumo

\newpage

%início abstract

\noindent{GUTO, A. C. N. R. \textbf{Título do Trabalho de Conclusão seu Moço}. 2013. N\degree 100 f. Tese (Bacharelado em Ciências da Computação) -- Universidade do Templo Unico do Saber, 2013.}
\vskip 1.0cm

\begin{center}
\textbf{ABSTRACT}
\end{center}
\vskip 0.5cm

\begin{noindent}
\noindent

\lipsum[4-5]

\vskip 10mm

\noindent \textbf{Key words: } 
\end{noindent}

%fim abstract

\clearpage

%lista de figuras
\renewcommand{\listfigurename}{\centerline{\normalsize \bf LISTA DE FIGURAS}}

\thispagestyle{empty}
\protect\thispagestyle{empty}
\listoffigures

\clearpage

%lista de tabelas
%\renewcommand{\listtablename}{\centerline{\normalsize \bf LISTA DE TABELAS}}

%\thispagestyle{empty}
%\protect\thispagestyle{empty}
%\listoftables

%\clearpage


%Sumário
\renewcommand{\contentsname}{\centerline{\normalsize \bf SUMÁRIO}}

\tableofcontents

\thispagestyle{empty}

\clearpage



\mainmatter

%\newcommand\nomecapitulo{}
%\renewcommand\chaptermark[1]{\renewcommand\nomecapitulo{#1}}

\fancypagestyle{plain}{
\fancyhf{} 
%\lhead{\nomecapitulo}
\fancyhead[R]{\thepage}
\renewcommand{\headrulewidth}{0pt}
\renewcommand{\footrulewidth}{0pt}}

\pagestyle{plain}

\setcounter{page}{14} 

\chapter*{Introdução}
\addcontentsline{toc}{chapter}{INTRODUÇÃO}
\label{intro}
%\lipsum[8-12]

% o comando  * retira o numero do capitulo - introdução e conclusão não são capítulos
\chapter[INTRODUÇÃO A FÍSICA ]{Introdução a Física }
\label{cap1}
%________________________________________________________________________________________________________________________


Por muito tempo pensou-se que a matéria apresentava-se em apenas três estados termodinâmicos, as fases sólida, líquida e gasosa, sendo essa idéia reforçada pelas experiências diárias, no qual é fácil reconhecer um composto qualquer estando em uma dessas fases, como por exemplo a água e o gelo \cite{sing,degennes}. Entretanto, em $1888$, o botânico austríaco \textsl{Reinitzer}, ao investigar esteres de colesterol, observou que esses compostos orgânicos ao sofrerem uma mudança de fase do estado líquido para o sólido apresentavam dois pontos de fusão, e não mudavam de fase de maneira ordinária. Em seu estudo, observou que a $145,5 \degree C$, o benzoato de colesterila fundia tornando-se um líquido com aspecto turvo e a $178,5 \degree C$ tornava-se um líquido claro. Um outro comportamento não comum foi observado ao se resfriar as amostras. Primeiramente o líquido claro apresentava um azul pálido antes de se tornar turvo e um azul violeta brilhante quando a amostra solidificava-se. \textsl{Reinitzer} então enviou suas amostras para o físico alemão \textsl{Lehmann}.

%Exemplo de Equação

\begin{equation}
X=X_{0}+n_{1}A_{1}+n_{2}A_{2}+n_{3}A_{3} 
\end{equation}

\noindent em que os $n_{i}$ são inteiros e os $A_{i}$ são os vetores de base ($i=1,2,3$). %com o comando \noindent, o Tex não cria um paragrafo na frase

%exemplo de figura -  \label{para poder fazer referencia durante o texto} e o \caption{legenda}
\begin{figure}[ht]
\centering
\caption{Estrutura cristalina no qual têm-se a rede (as linhas do cubo) e a base (os átomos constituintes, representados pelos círculos nos vértices).}
\label{figcrist}
\end{figure}




%________________________________________________________________________________________________________________________

\section{Seção}
%________________________________________________________________________________________________________________________



%________________________________________________________________________________________________________________________

\subsection{Subseção}

%________________________________________________________________________________________________________________________



%________________________________________________________________________________________________________________________

\subsubsection{Subseção da Subseção}
%________________________________________________________________________________________________________________________

\chapter[EM CAIXA ALTA]{Em caixa alta\\ Diferença do título e do que aparece no sumário}
%\thispagestyle{empty}
%\chapter[CRISTAIS LÍQUIDOS NEMÁTICOS]{Cristais Líquidos Nemáticos}
\label{CLN}








%______________________________________________________________________________________________________________
%______________________________________________________________________________________________________________
%______________________________________________________________________________________________________________
%______________________________________________________________________________________________________________
%______________________________________________________________________________________________________________

\chapter[GEOMETRIA DIFERENCIAL]{Geometria Diferencial}
\label{capII}
%\lipsum[19-24]


Ao se olhar para um objeto qualquer, as linhas que o compõem (delimitam ou constroem) são reconhecidas e associadas a representações geométricas, tais como, linhas curvas ou retas. De maneira intuitiva, define-se uma linha curva ou reta, em relação a um dado referencial\footnote{\textsl{Nesse contexto referencial é entendido simplesmente como algo a ser utilizado para comparação entre os objetos geométricos.}}.

%exemplo de nota de rodapé - \footnote{}
\chapter[TENSORES]{Tensores}
\label{tensor}


















\chapter[AS CONSTANTES ELÁSTICAS ]{As Constantes Elásticas}
\label{cap7}

Em meados da década de 1980, S. Hess juntamente com colaboradores apresentaram um trabalho, no qual expuseram que propriedades como a  elasticidade, a viscosidade, entre outras, poderiam ser obtidas a partir de uma transformação afim (ponto a ponto) de um potencial esférico, de um líquido isotrópico,  para um um potencial de interação elipsoidal de um cristal líquido \cite{baalss}. Anos mais tarde, em meados de 2007, M. Simões e colaboradores, propuseram um método que considera tal deformação do potencial como dependente localmente da posição \cite{simoes}. 


%exemplo de citação de referencia no modelo de artigo \cite{label da referencia}

%exemplo de tabela



\begin{table}[htbp]
  \centering
  \caption{CB8}
    \begin{tabular}{cccccc}
    \hline
    K11   & K22   & K33   & Temperatura Reduzida & R21 (K22/K11)  & R31 (K33/K11) \\
    \hline
    1,06  & 0,35  & 1,25  & 0,9995 & 0,33019 & 1,17925 \\
    1,51  & 0,45  & 1,82  & 0,997 & 0,29801 & 1,2053 \\
    1,58  & 0,48  & 1,9   & 0,996 & 0,3038 & 1,20253 \\
    1,71  & 0,52  & 2,09  & 0,9935 & 0,30409 & 1,22222 \\
    1,98  & 0,55  & 2,48  & 0,99  & 0,27778 & 1,25253 \\
    2,27  & 0,63  & 2,84  & 0,9845 & 0,27753 & 1,2511 \\
    \hline
    \end{tabular}%
  \label{tabcb8}%
\end{table}%


%\chapter[MAIS UM CAPITULO ]{Mais um Capitilo}
\label{relgel}





\chapter*{Considerações Finais e Conclusões}
\addcontentsline{toc}{chapter}{CONSIDERAÇÕES FINAIS E CONCLUSÕES}





\cleardoublepage
\phantomsection
\renewcommand{\bibname}{Referências}
\begin{thebibliography}{99}
\addcontentsline{toc}{chapter}{REFERÊNCIAS}

%____________________________________________________________________________________________________
%Introdução
%____________________________________________________________________________________________________

\bibitem{art1} ZUREK W. H., Nature 317, 505 (1985); 

\bibitem{art2} CHUANG I., DURRER R., TUROK N., YURKE B., Science 251, 4999, 1336 (1991);
 
%\bibitem{art3} KIBBLE T. W. B., Physica C 369, 87 (2002); 

\bibitem{art4} KIBBLE T. W. B., Physics Today september, 47 (2007);
 
%\bibitem{art5} KIBBLE T. W. B., Phil. Trans. R. Soc. 366, 2793 (2008); 

\bibitem{art6} RAY R., SRIVASTAVA A. M., Physical Review D 69, 103525 (2004); 

\bibitem{art7} MONACO R., MYGIND J., AAROE M., RIVERS R. J., KOSHOLETS V. P., PRL 96, 
180604 (2006); 

\bibitem{simoes} SIMOES M., CAMPOS A., BARBATO D.,  Phys. Rev. E \textbf{75}, 061710 (2007);

\bibitem{baalss} BAALSS D., HESS S., Phys. Rev. Lett. \textbf{57}, 86 (1986);

%____________________________________________________________________________________________________
%Capítulo 1     CRISTAIS LÍQUIDOS
%____________________________________________________________________________________________________


\bibitem{sing} Singh S., \textit{Liquid Crystals - Fundamentals}, \textbf{World Scientific Publishing}, New Jersey, Lindon, Singapore, Hong Kong (2002).

\bibitem{degennes} DE Gennes P. G., Prost J, \textit{The Physics of Liquid Crystals}, \textbf{Clarendon Press, Oxford}, (1993).

\bibitem{chandra} Chandrasekhar S., \textit{Liquid Crystals}, 2nd Ed. \textbf{Cambridge University Press},  (1992).

\bibitem{kittel} Kittel, C., \textit{Introduction to Solid State Physics}.7.ed New York: \textbf{John Wiley $\&$ Sons}, Inc., (1996).

\bibitem{Figueiredo} Figueiredo, A. M. N.; Salinas, S. R. A.; \textit{The Physics of Lyotropic Liquid Crystals}, Oxford University Press, New York (2005).

\bibitem{oswaldo}P. Oswald., P. Peranski, \textit{Nematic and Cholesteric Liquid Cystals} (Taylor \& Francis Group, 2005)

 


%____________________________________________________________________________________________________
%Capítulo 2  GEOMETRIA DIFERENCIAL
%____________________________________________________________________________________________________

%RUSSOS   

\bibitem{stuik} Struik, D. J.; \textit{Lectures on Classical Differential Geometry}, 1 ed. Cambridge: \textbf{Addison - Wesley Press,} INC., (1950)

\bibitem{stoker} Stoker, J. J.; \textit{Differential Geometry}, 1.ed New York: \textbf{John Wiley $\&$ Sons}, Inc., 1969.

\bibitem{landau} Landau, L., Lifchitz, E., \textit{Theory of Elasticity,}, 3nd Ed. \textbf{Heinenmann},  (1986).

\bibitem{lugo} Lugo, G.; \textit{Defferential Geometry in Physics - Lectures Notes}, \textbf{Departament of Mathematical Science}, University of North Carolina at Wilmington (1995 - 1998)

\bibitem{spivak} Spivak, M.; \textit{A Comprehensive Introduction to Differential Geometry}, Vol. 2, 3rd Edition, Houston, Texas: \textbf{Publish or Perish}, INC.,(1999)

\bibitem{weinberg} WEINBERG S., \textit{Gravitation and Cosmology: Principles and Applications of the General 
Theory of Relativity}, \textbf{John Wiley and Sons}. (1972); 

%LEVICIVITA  SPIVAK  MANFREDO PERDIGÃO 


%____________________________________________________________________________________________________
%Capítulo 4  TENSORES
%____________________________________________________________________________________________________


%ARFKEN BUTKOV DINVERNO MOULD  







\end{thebibliography}


\cleardoublepage
\phantomsection
\appendix
\addcontentsline{toc}{chapter}{APÊNDICES}
\chapter*{Apêndices}
\cleardoublepage

%\include{A_DataInformation}
%________________________________________________________________________________________________________________________________________________________________

\chapter{Propriedades das formas Fundamentais}

%________________________________________________________________________________________________________________________________________________________________

\section{Pontos de Máximo e Mínimo}

%________________________________________________________________________________________________________________________________________________________________

A curvatura normal e a curvatura gaussiana possuem o sinal dependente apenas da segunda forma fundamental, visto que, a primeira forma quadrática é positivo definida, pois, como mostrado anteriormente, ela é $\ud l^{2}$ (módulo positivo para números reais). 



%________________________________________________________________________________________________________________________________________________________________
%|||||||||||||||||||||||||||||||||||||||||||||||||||||||||||||||||||||||||||||||||||||||||||||||||||||||||||||||||||||||||||||||||||||||||||||||||||||||||||||||||||||||||||||||||||||||||||||||||||||||||||||||||||||||||||||||||||||||||||||||||||||||||||||||||||||||||||||||||||||||||
%|||||||||||||||||||||||||||||||||||||||||||||||||||||||||||||||||||||||||||||||||||||||||||||||||||||||||||||||||||||||||||||||||||||||||||||||||||||||||||||||||||||||||||||||||||||||||||||||||||||||||||||||||||||||||||||||||||||||||||||||||||||||||||||||||||||||||||||||||||||||||


\chapter{As equações de Gauss- Weingarten Novamente}
\label{labeldocapitulo}


%|||||||||||||||||||||||||||||||||||||||||||||||||||||||||||||||||||||||||||||||||||||||||||||||||||||||||||||||||||||||||||||||||||||||||||||||||||||||||||||||||||||||||||||||||||||||||||||||||||||||||||||||||||||||||||||||||||||||||||||||||||||||||||||||||||||||||||||||||||||||||
%|||||||||||||||||||||||||||||||||||||||||||||||||||||||||||||||||||||||||||||||||||||||||||||||||||||||||||||||||||||||||||||||||||||||||||||||||||||||||||||||||||||||||||||||||||||||||||||||||||||||||||||||||||||||||||||||||||||||||||||||||||||||||||||||||||||||||||||||||||||||||

Nesse capítulo iremos escrever as equações de Gauss e Weingarten em uma notação mais compacta, possibilitando introduzir o símbolo de \textsl{Riemann} e dessa maneira reescrever a curvatura de Gauss.

%________________________________________________________________________________________________________________________________________________________________


\section{As Equações de Gauss}

%________________________________________________________________________________________________________________________________________________________________




\backmatter
%\include{C_Anexos} %deixei como exemplo exatamente como usei. Note que sai certinho no sumário.

\thispagestyle{empty}
\begin{center}
{\color{white} Minha página em branco} %contracapa
\end{center}


\end{document}
